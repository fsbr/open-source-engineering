\documentclass[10pt,a4paper]{article}
\usepackage[utf8]{inputenc}
\usepackage{amsmath}
\usepackage{amsfonts}
\usepackage{amssymb}
\usepackage{graphicx}
\author{Thomas Fuller}
\title{Nulling Filter}
\usepackage{hyperref}
\hypersetup{
    colorlinks=true,
    linkcolor=blue,
    filecolor=magenta,
    urlcolor=cyan,
}



\begin{document}
\maketitle
\section*{Introduction}
In this lab we're going to be developing a nulling filter to filter out noise in a given black and white image.  A nulling filter
\\ \\
\textbf{NOTE:} In this particular image a white pixel takes on a value of 255, and a black pixel takes on a value of 0.  
\\ \\
\textbf{NOTE:} In MATLAB, the datatype unit8 obeys integer division rules, so for example if a=4 is of the unsigned integer datatype, then a/2 returns a value of 2, but rounding occurs if a would be a decimal, so a/3 returns a value of 1, even though as humans if we divided 4/3 we get a=1.3333.  To fix this you can cast values to "double" as you see appropriate for this lab.  
\\ \\
\section*{Part 1}
The lab instructions use the MATLAB commands but you're definitely welcome to use the Python opencv equivalents if you want to be a boss. 
1.  Using the matlab \texttt{imread}, and \texttt{imshow} commands, read in the attached image ``homework.png".  We will use the \texttt{imshow} command with the open and square bracket option to automatically scale our image throughout this lab.  
\\ \\
2. Create a figure with two subplots.  The first subplot should contain the values of each pixel in the 99th ROW of the image, and have labels ``index (n)" and ``pixel value" on the x and y axis respectively.  The second subplot should be a graph of the ``index (n)" on the horizontal axis, but the y axis will be the data in the 99th row passed through an 11 point moving average filter.
\\ \\
3.  \textbf{How to achieve the 11 point moving average filter:}  You can implement this filter by convolving (\texttt{conv(a,b)}) the data in the 99th row of the homework image with the coefficients $b_{k} = \frac{1}{L}\{1 1 1 ... 1 1 1\}$, which is a vector of ones, of length L, and $L=11$.  

\section*{Part 2}
Repeat the steps of part 1, except implement the running average filter over the entire image, and display the output on a figure with two subplots, showing what the image looked like before and also after.  

\section*{Part 3}
%%%% i think i need to like vary the data somewhat.
I've corrupted the image such that for each row in the image, the corrupted output is
\begin{equation}
x_{1}[n] = 128 + 128\cos(\frac{2 \pi}{11}n) + x[n]
\end{equation}
\\ \\
1.  Using the \texttt{diric} command in matlab, or the equation
\begin{equation*}
D_{L}(\hat{\omega}) = \frac{1}{L}\frac{\sin(L \hat{\omega}/2)}{\sin(\hat{\omega}/2)}
\end{equation*}
Plot the magnitude part of the frequency response, with ``normalized radian frequency" on the horizontal axis, and $|H(e^{j \hat{\omega}})|$ on the vertical axis.

\section*{Part 4}
Pass the filter over the entire corrupted image and explain why it works, using the plot of the Dirichlet function in part 3 as a reference. Also explain some of the side effects of this filter.  

\section*{Part 5}
Answer the following questions:
\\ \\
1.  What is the relationship between $\hat{w}$ and $w$?
\\ \\
2.  State the Nyquist Sample theorem and explain why sample rate is not a factor in this particular example. Also state when sample rate WOULD be a factor.
\\ \\
3.  If we were attempting to implement this filter on a continuous system, and we sampled at a rate of $f_{s} = 1000$(samples/second), what frequency (in Hz), would be filtered out of the signal?

\section*{Inspirational Quote}
You can do it!
\end{document}

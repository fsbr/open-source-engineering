\documentclass[10pt,a4paper]{article}
\usepackage[utf8]{inputenc}
\usepackage{amsmath}
\usepackage{amsfonts}
\usepackage{amssymb}
\usepackage{graphicx}
\author{Thomas Fuller}
\title{Enrichment Problems}
\usepackage{hyperref}
\hypersetup{
    colorlinks=true,
    linkcolor=blue,
    filecolor=magenta,
    urlcolor=cyan,
}
\begin{document}
\maketitle
\section*{Introduction}
I want to give you guy some problems that help in our fundamental understanding of discrete time systesm.
\section*{FIR Filter}
An LTI system has a frequency response
\begin{equation*}
H(e^{j\hat{\omega}} = (1 -e^{j\pi/2}e^{-j\omega})(1 -e^{-j\pi/2}e^{-j\omega})(1 +e^{-j\omega})
\end{equation*}
The input to the system is 
\begin{equation*}
x[n] = 5 + 20\cos(0.5 \pi n + 0.25 \pi) + 10\delta[n-3]
\end{equation*} 
Find the output of the system

\section*{Another one}
An LTI system has the difference equation
\begin{equation*}
y[n] = -x[n] + 2x[n-2] - x[n-4]
\end{equation*}
I. Find the impulse response h[n]  and plot it
\\ \\
II. Determine an equation for the frequency response $H(e^{j\hat{\omega}})$ and express it in the form
\begin{equation*}
R(e^{j\hat{\omega}})e^{-j \hat{\omega}n_{0}}
\end{equation*}
where $n_{0}$ is an integer.
\section*{Sample and Reconstruct}
The signal $x(t)$ is to be reconstructed by directly putting an ideal continuous to discrete converter in cascade with an ideal digital to discrete converter, each samples at a rate of $f_{s} = 500$(samples/sec). $x(t)$ is defined as follows. 
\begin{equation*}
x(t) = 5\cos(200 \pi t) + 10 \cos(800 \pi t + \frac{\pi}{3}) + 2\cos(2200 \pi t + \frac{\pi}{6})
\\ \\
\end{equation*}
\\ \\
I.  Draw the block diagram of the entire process
\\ \\
II.  Determine the output function $y(t)$.  
\section*{Discrete Fourier Transform}
Find the N point DFT of $u = [\sin(ja)]_{j=0}^{N-1}$, in which $N$ is  a positive integer and $a$ is a given complet number.  To avoid trivialities suppose $a$ is not an integer multiple of $\pi$. 
\end{document}






